%%%%%%%%%%%%%%%%%%%%%%%%%%%%%%%%%%%%%%%%%
% Medium Length Professional CV
% LaTeX Template
% Version 2.0 (8/5/13)
%
% This template has been downloaded from:
% http://www.LaTeXTemplates.com
%
% Original author:
% Trey Hunner (http://www.treyhunner.com/)
%
% Important note:
% This template requires the resume.cls file to be in the same directory as the
% .tex file. The resume.cls file provides the resume style used for structuring the
% document.
%
%%%%%%%%%%%%%%%%%%%%%%%%%%%%%%%%%%%%%%%%%

%----------------------------------------------------------------------------------------
%	PACKAGES AND OTHER DOCUMENT CONFIGURATIONS
%----------------------------------------------------------------------------------------

\documentclass{resume} % Use the custom resume.cls style

\usepackage[left=0.65in,top=0.5in,right=0.65in,bottom=0.5in]{geometry} % Document margins
\usepackage{CJK}
\usepackage{url}
\usepackage{graphicx}
\usepackage[colorlinks=true,linkcolor=black,urlcolor=black,hyperindex,CJKbookmarks]{hyperref}

\begin{CJK}{UTF8}{gkai}

\name{寿林钧} % Your name
\address{{Date: 1989-10-21} \hfill {Email: shoulinjun@126.com}}
\address{{Blog: {\url{http://blog.csdn.net/shoulinjun}}} \hfill {Phone: (+86)131-4692-5206}}

%----------------------------------------------------------------------------------------
%	EDUCATION SECTION
%----------------------------------------------------------------------------------------
\begin{document} 
\begin{rSection}{求职意向} 
 游戏研发工程师 R \& D (2015年7月毕业) 
\end{rSection} 

\begin{rSection}{教育背景} 
\begin{itemize} 
  \setlength{\itemsep}{0pt} %条目间距 
  \setlength{\parsep}{0pt} %段落间距 
  \setlength{\parskip}{0pt} %列表到上下文的垂直距离 
\item {\bf 中国科学院计算技术研究所 高性能计算机研究中心} \hfill {2012.09 - 2015.07} \\ 
工学硕士   计算机应用技术 (保送) \hfill 研究方向: 数值模拟算法研究 
\vspace{1mm}
\item {\bf 上海大学 理学院} \hfill {2008.09 - 2012.06} \\ 
理学学士   理科基地班  \hfill 绩点: {\bf 3.80/4.0} 专业排名: {\bf 1/52}\\
以{\bf专业第一名},{\bf学院第二名}(共{\bf297}人)的成绩保送到中国科学院大学 {\hfill}成绩:{\bf 前 $1\%$}
\end{itemize}
\end{rSection}

%----------------------------------------------------------------------------------------
%	WORK EXPERIENCE SECTION
%----------------------------------------------------------------------------------------

\begin{rSection}{项目经历}

\begin{rSubsection}{新型数值模拟算法研究及 pGFEM 软件的实现}{2013.07 - 2014.05}{\bf 核心开发人员}{\bf 中科院计算所}

\item {简介:}
\setlength{\itemsep}{0pt} %条目间距
\setlength{\parsep}{0pt} %段落间距
\setlength{\parskip}{0pt} %列表到上下文的垂直距离
\begin{itemize}
\setlength{\itemsep}{0pt} %条目间距
\setlength{\parsep}{0pt} %段落间距
\setlength{\parskip}{0pt} %列表到上下文的垂直距离
  \item 提出了一种新型有限元数值算法,可有效用于桥梁受载、高速运动高铁车厢受力、导弹打靶等问题的模拟,以及电影特效制作中的场景动态仿真。
  \item 该算法通过将实际物体离散化为网格或者质点,用网格或质点的速度、压强、密度等变量(位置和时间的函数)模拟实际的运动。根据能量泛函最小化得到关于节点位置与时间的非线性方程组,进而用牛顿-拉夫森迭代方法求解。
\end{itemize}

\item 职责:
\setlength{\itemsep}{0pt} %条目间距
\setlength{\parsep}{0pt} %段落间距
\setlength{\parskip}{0pt} %列表到上下文的垂直距离
\begin{itemize}
\setlength{\itemsep}{0pt} %条目间距
\setlength{\parsep}{0pt} %段落间距
\setlength{\parskip}{0pt} %列表到上下文的垂直距离
    \item 独立完成整个算法的实现与测试;
    \item 负责整个软件的整体框架设计;
    \item 负责代码计算模块的性能优化;
    \item 设计面向对象的数据结构,提高代码重用性,便于用户进行二次开发,不必考虑算法细节。
\end{itemize}
\item 成果: 
\begin{itemize}
\setlength{\itemsep}{0pt} %条目间距
\setlength{\parsep}{0pt} %段落间距
\setlength{\parskip}{0pt} %列表到上下文的垂直距离
\item 本算法的实验数值精度相较于主流方法提高了{\bf 10\%}以上;
\item 本算法克服了传统算法在超大变形问题模拟中的网格重构问题,大大减少了计算量;
\item 完成 pGFEM 软件代码的编写,并已经与某电影制作公司合作进行场景动态仿真。
\end{itemize}

\item 关键词: \\
  C++ \quad 数值计算 \quad 代码优化 \quad 面向对象

\end{rSubsection}

%------------------------------------------------

\begin{rSubsection}{C++经典开源代码学习}{2013.09 - 2014.04}{}{}
\item C++ STL标准程序库
\setlength{\itemsep}{0pt} %条目间距
\setlength{\parsep}{0pt} %段落间距
\setlength{\parskip}{0pt} %列表到上下文的垂直距离

\begin{itemize}
\setlength{\itemsep}{0pt} %条目间距
\setlength{\parsep}{0pt} %段落间距
\setlength{\parskip}{0pt} %列表到上下文的垂直距离
  \item 熟悉内存配置器、容器与算法三部分,并自行实现了vector、list、deque、string等容器与算法;
  \item 熟悉 Copy On Write、智能指针、引用计数、Traits、垃圾回收等技术,并将其用到项目之中;
\end{itemize}

\item LevelDB 数据库
\begin{itemize}
\setlength{\itemsep}{0pt} %条目间距
\setlength{\parsep}{0pt} %段落间距
\setlength{\parskip}{0pt} %列表到上下文的垂直距离
  \item 重点研读 Memtable、内存分配器等部分;
  \item 对 NoSQL数据库的设计与实现有一定的了解。
\end{itemize}
\item 关键词:\\ 
  STL源码 \quad Traits  \quad 智能指针 \quad {Copy On Write}  \quad NoSQL 

\end{rSubsection}

\end{rSection}

%----------------------------------------------------------------------------------------
%	REWARDS SECTION
%----------------------------------------------------------------------------------------

\begin{rSection}{所获奖励}

\begin{tabular}{ @{} >{}l @{\hspace{2ex}} lll }
  2008/09/10 & {\bf特等奖学金}三次({\bf 前 1\%}) & 2009/10 & 优秀学生两次  \\ 
  2010 & {\bf国家奖学金}({\bf 前 1\%})& 2012  & 上海市优秀毕业生({\bf 前 1\%}) \\
  2012 & 中国科学院大学三好学生 (前 5\%)     \\
\end{tabular}

\end{rSection}

%----------------------------------------------------------------------------------------
%	TECHNICAL STRENGTHS SECTION
%----------------------------------------------------------------------------------------

\begin{rSection}{个人技能}

\begin{tabular}{ @{} >{}l @{\hspace{5ex}} l }
  算法能力 & 良好的算法和数据结构基础 \\
  编程能力 & 熟悉 linux 下 C/C++ 编程,了解面向对象基本思想\\
  英语水平 & 雅思 {\bf 7.0},六级 {\bf 617} 分\\
  综合能力 & 极强的学习能力,喜欢钻研问题
\end{tabular}

\end{rSection}

%----------------------------------------------------------------------------------------
%	EXAMPLE SECTION
%----------------------------------------------------------------------------------------

%\begin{rSection}{Section Name}

%Section content\ldots

%\end{rSection}

%----------------------------------------------------------------------------------------

\end{CJK}
\end{document}
