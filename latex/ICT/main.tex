%\documentclass[slidestop,compress,mathserif,blue]{beamer}
\documentclass[mathserif]{beamer}
%\usepackage[noindent,nofonts]{ctexcap} 
\usepackage{CJK} 
%\setCJKmainfont[BoldFont=STHeiti]{STXihei} 
%\setCJKmainfont[BoldFont=STZhongsong]{STZhongsong} 
\usepackage{beamerthemeshadow}
\usepackage{beamerthemesplit}
%\usetheme{shadow}
\usecolortheme{default}
\setbeamertemplate{footline}[frame number]
\useinnertheme[shadow=true]{rounded}
%\setbeamertemplate{footline}{\insertframenumber/\inserttotalframenumber}
%\useoutertheme{infolines}
%\setbeamertemplate{headline}{} % removes the headline that infolines inserts

%\usetheme{boxes}
%\usepackage{amsmass}
%\usepackage{amssymb,amsfonts,url}

\usepackage{algorithm}
\usepackage{algorithmic}

\usepackage{graphicx}
\graphicspath{{Problems/}}

\usepackage{tikz}
\usepackage{verbatim}
\usepackage{pgfplots}
\usepackage{verbatim}
\usetikzlibrary{arrows,shapes}
\usepackage{color}

\usepackage{multirow}
\usetheme{Frankfurt}
\usecolortheme{default}
\setbeamertemplate{frametitle}[default][left]

\usepackage{graphicx}
%\graphicspath{{./fig/}}
\usepackage{float}
%\usepackage{color}

\usepackage{cases}
\usepackage{mathrsfs}

\usepackage{url}

\usepackage{booktabs}

\newcommand{\rmnum}[1]{\romannumeral#1}
\newcommand{\Rmnum}[1]{\uppercase\expandafter{\romannumeral#1}}

%设置图片标题的字体大小
\usepackage{caption}
\captionsetup{font={scriptsize}}

\begin{document}
\begin{CJK}{UTF8}{gkai}

\title{面向千万亿次计算的高可扩展大型非结构化网格有限元算法研究}
\author{导师:田荣 \\ 学生:寿林钧} 
\institute{中国科学院计算技术研究所\\高性能计算机研究中心} 
\date{\today} 

\pgfdeclareimage[height=0.7cm]{logo}{./ljshou_figures/ICT.png}
\logo{\pgfuseimage{logo}}
   
\begin{frame}
\titlepage
\begin{figure}
	\includegraphics[width=2.5in]{./ljshou_figures/UCAS3.png}
\end{figure}
\end{frame}

\begin{frame}
   \frametitle{提纲}
   \tableofcontents
        
        \begin{itemize}
            \item 研究背景
                \begin{itemize}
                \item "千万亿次计算"现状
                \item 主要技术挑战
                \end{itemize}
            \item 相关研究
                \begin{itemize}
                \item 协调网格生成
                \item 组合网格粘结
                \end{itemize}
            \item 研究内容
    		  \begin{itemize}
    		  	\item 自由网格有限元
            \item 非线性大变形程序开发
    		  \end{itemize}
            \item 算例
            \item 参考文献
        \end{itemize}
\end{frame}


%%%%%%%%%%%%%%%%%%%%%%%%%%%%%%%%%%%%%%%%%%%%%%%%%%%%%%%%%%%%%%%%%%%%%%%%%%%%%%%%%%%%%%
%%%%%%%%%%%%%%%%%%%%%%%%------------BACKGROUND--------------------------------%%%%%%%%
%%%%%%%%%%%%%%%%%%%%%%%%%%%%%%%%%%%%%%%%%%%%%%%%%%%%%%%%%%%%%%%%%%%%%%%%%%%%%%%%%%%%%%
%\section{研究背景}
%\subsection{“千万亿次计算”现状}
\begin{frame}
\frametitle{研究背景: “千万亿次计算”}
  \begin{itemize}
\item “千万亿次计算”现状
   \begin{itemize}
	\item Top 500列表最新的统计显示,理论峰值1Petaflops以上的超级计算机已经达到55台
   \end{itemize} 
   
\item “千万亿次计算”的关键问题
   \begin{itemize}
	\item 程序的可扩展能力
	\item 程序的可容错能力
	\item 浮点效率
	\item 异构协同
   \end{itemize}
  \end{itemize}
\end{frame}

%\subsection{主要技术挑战}
\begin{frame}
	\frametitle{研究背景:主要技术挑战(1)}
	大型复杂非结构化网格有限元模型的建立
	\begin{itemize}
	  \item 大规模有限元的成功应用仍以结构化网格为主,非结构化网格应用严重受到网格生成困难的制约
	  \item {\color{red}\bf 对于非结构化网格,在网格上所花费的时间约占整体计算时间的90\%}
	  \item IO数据量的挑战
	  \begin{itemize}
		\item 100亿自由度网格,输入数据 ~678GB(二进制格式)
	  \end{itemize}
	\end{itemize}
		\begin{figure}
			\centering
			\includegraphics[width=3in]{./ljshou_figures/model_io.png}
		\end{figure}
\end{frame}

	\begin{frame}
	\frametitle{研究背景:主要技术挑战(2)}
	\begin{columns}
		\begin{column}[pos]{0.4\textwidth}
			\begin{figure}
				\includegraphics[width=1.2in]{./ljshou_figures/traditional_fem.eps}
			\end{figure}
		\end{column}
		
		\begin{column}[pos]{0.6\textwidth}
			\begin{itemize}
	  			\item 大多历史遗产代码(特别是国外商业程序)大多诞生于PC时代,算法架构设计没有、也无法考虑目前的海量并行需求
	  		\begin{itemize}
				\item 面对未来海量并行,架构性矛盾突出,这是目前商业程序不可扩展的根本原因
	  		\end{itemize}
	  		\item 面向P级E级计算,应用数学算法的设计应该考虑并行计算本身的特点,如对并行性,数据移动复杂性,甚至容错性的重视
	\end{itemize}
		\end{column}
	\end{columns}
	\begin{block}{}
		\color{red}\bf 历史遗产并行架构在前、后处理上的串行瓶颈
	\end{block}
\end{frame}

%\section{相关研究}
%\subsection{协调网格生成}
	\begin{frame}
	\frametitle{相关研究:协调网格生成}
	\begin{itemize}
	  \small
	  \item Conforming assembly meshing (CUBIT (Blacker,1994); Assembly meshing (Tautges,2001); Tolerant imprinting (Clark et al., 2008); Mesh matching (Staten et al.,2008)(协调的组合网格生成)
		\begin{itemize}
		\scriptsize
		\item Globally conforming
            \item {\color{red}Propagation of mesh and refinement across shared surfaces}
            \item {\color{red}Dirty geometry}: overlap and gap between adjacent CAD model of parts; Without clear definitions of the interfaces between adjacent volumes it is very difficult to generate a conformal assembly mesh (Clark et al., 2008). 
	  	\end{itemize}
	  	\begin{figure}
	  		\includegraphics[width=2in]{./ljshou_figures/mesh_matching.png}
	  	\end{figure}
	\end{itemize}
\end{frame}

%\subsection{组合网格粘结}
\begin{frame}
	\frametitle{相关研究:组合网格粘结}
	\begin{itemize}
	  %\small
	  \item Non-conforming assembly mesh gluing using Lagrange multipliers(非协调组合网格的拉格朗日乘子粘接)
	  \begin{itemize}
	  	\scriptsize
		\item Mortar element Methods (Bernardi, Maday and Patera(1990); Belgacem and Maday(1997))
		\item FETI methods (Farhat and Roux(1991); Farhat and Geradin(1992)) 
		\item MPCs, tied contact, and gap elements in commercial software
	  \end{itemize}
	  \item 缺点: 
	    \begin{itemize}
	    \item 在变分意义上满足子网格界面处的连续性要求
	    \item 方法引入的新的自由度(拉格朗日乘子),容易导致集成的{\color{red}刚度矩阵具有非正定性},这会对方程的求解带来困难。
	    \end{itemize}
	\end{itemize}
\end{frame}



%%%%%%%%%%%%%%%%%%%%%%%%%%%%%%%%%%%%%%%%%%%%%%%%%%%%%%%%%%%%%%%%%%%%%%%%%%%%%%%%%%%%%%
%%%%%%%%%%%%%%%%%%%%%%%%------------INTERESTS---------------------------------%%%%%%%%
%%%%%%%%%%%%%%%%%%%%%%%%%%%%%%%%%%%%%%%%%%%%%%%%%%%%%%%%%%%%%%%%%%%%%%%%%%%%%%%%%%%%%%	
%\section{研究内容}
%\subsection{自由网格有限元}
	\begin{frame}
	\frametitle{研究内容:自由网格有限元}
		\begin{itemize}
			\item 组合化网格优点: 
				\begin{itemize}
					\item 网格生成简单
					\item 网格局部加密、重构等操作更加容易施加
					\item 前后处理具有高度并行性
				\end{itemize}
				\begin{figure}
    					\centering
    					\includegraphics[width=2in]{./ljshou_figures/fem_compare.eps}
    					\caption{并行有限元分析:(a)协调网格 (b)组合网格}
		        \end{figure}
		\end{itemize}
	\end{frame}	
	
	\begin{frame}{自由网格有限元:组合网格划分}
    		\begin{itemize}
    			\item 组合化网格在网格生成上有天生的并行性。用一个共享内存的刀片(40CPUs),并行地对8669块子部件进行网格划分
    		    \item 划分得到的组合网格由7,493,401个四面体单元组成。而整个网格划分过程仅耗时30分钟
    		\end{itemize}
    		\begin{figure}
    		\centering
    		\includegraphics[width=4.0in]{./ljshou_figures/model_generation.PNG}
    		%\caption{Caption}
    		\end{figure}
    		\tiny Tian R*, Wu ZD, Wang CW, Scalable FEA on non-conforming assembly mesh. Computer Methods in Applied Mechanics \& Engineering. 2013; 266: 98-111
	\end{frame}

	\begin{frame}
		\frametitle{自由网格有限元:技术难点}
				\begin{itemize}
					\item 如何将网格“粘结”起来?保证内部区域的连续性:
					\begin{equation*}
						{\bf u}^i = {\bf u}^j \text{ on } \Gamma_{ij} = \Omega_i \cap \Omega_j
					\end{equation*}
				\end{itemize}
				\begin{figure}
					\centering
					\includegraphics[width=2in]{./ljshou_figures/combined_mesh.eps}
					\caption{组合网格界面非连续性}
				\end{figure}
	\end{frame}
			

%%%%%%%%%%%%%%%%%%%%%%%%%%%%%%%%%%%%%%%%%%%%%%%%%%%%%%%%%%%%%%%%%%%%%%%%%%%%%%%%%%%%%%
%%%%%%%%%%%%%%%%----------------ALGORITHM-------------------------------------%%%%%%%%
%%%%%%%%%%%%%%%%%%%%%%%%%%%%%%%%%%%%%%%%%%%%%%%%%%%%%%%%%%%%%%%%%%%%%%%%%%%%%%%%%%%%%%  		
	\begin{frame}{进展之一:自由网格组合算法}
	    网格耦合算法: 数学插值与几何网格拓扑分离
    		\begin{enumerate}
        		\item 算法1:背景网格法
                \item 算法2:全域无网格粘结
                \item 算法3:界面无网格插值
    		\end{enumerate}
    		\begin{columns}
    		    %the left column
    			\begin{column}[pos]{0.5\textwidth}
    				\begin{figure}
    					\includegraphics[width=1.3in]{./ljshou_figures/conforming_fem.eps}
    					\includegraphics[width=1.3in]{./ljshou_figures/math_glue.eps}
    					\caption{算法1:\color{red}几何网格与插值网格分离}
    				\end{figure}
        			\begin{itemize}
        			    \item {\color{red}非协调}组合网格:材料边界描述、数值积分、后处理
                    \item {\color{red}规则}数学网格:高质量数学插值; 避免网格扭曲
        			\end{itemize}
    			\end{column}

			%the right column
    			\begin{column}[pos]{0.5\textwidth}
    				\begin{figure}
                	\centering
                	\includegraphics[width=1.5in]{./ljshou_figures/interface_glue2.eps}
                	\caption{算法2, 3}
                \end{figure}
                \begin{itemize}
                \item \color{blue}独立网格的无网格粘结技术(高可扩展性,不增加方程求解困难)
                \end{itemize}
                
    			\end{column}
	    \end{columns}
	\end{frame}

	\begin{frame}
		\frametitle{背景网格法: step TL算法}
		\begin{columns}
		    %the left column
    			\begin{column}[pos]{0.4\textwidth}
    				\begin{figure}
    					\centering
    					\includegraphics[width=2.0in]{./ljshou_figures/step.png}
    					\label{fig:1}
    					\caption{Step完全拉格朗日方法}
    			    \end{figure}
    			    \begin{figure}
    					\centering
    					\includegraphics[width=2.0in]{./ljshou_figures/remesh.eps}
    					\label{fig:2}
    					\caption{$x^n$构型的更新}
    			    \end{figure}
    			\end{column}
    			
    			%the right column
    			\begin{column}[pos]{0.6\textwidth}
    			    \footnotesize
    			    \begin{itemize}
    			    \item 当前构型下的强形式:
            			\begin{equation*}
            				\begin{cases}
            					\nabla {\bf \sigma} + {\bf f} = 0& \text{in }\Omega\\
            					{\bf u=\bar{u}}& \text{on }\Gamma_u\\
            					{\bf n}\cdot{\bf \sigma} = {\bf t}& \text{on }\Gamma_t
            				\end{cases}
            			\end{equation*}
            		\item 中间构型的引入:
            		    \begin{equation*}
            		        \bf{F=\Delta F \cdot{F^n} \quad  F^n=\frac{\partial{x^n}}{\partial{X}}}
            		    \end{equation*}
            		\item 离散化的材料建立在结构化的背景网格$V$上, $V \supseteq \Omega$
            			\begin{equation*}
            				{\bf x} = \sum_{I=1}^{n}N_I{\bf x}_I(t), \quad supp(N) = \{{\bf x} \in V \supseteq \Omega\}
            			\end{equation*}
            	    %\item 虚功方程及内力表达式:
            	    %    \begin{equation*}
            	    %    \delta W()
            	    %    \end{equation*}
            	    %\item 总体非线性方程:
    			    \end{itemize}
            			
    			\end{column}
		\end{columns}
	\end{frame}	


        \begin{frame}{进展之二:非线性大变形程序统一实现}
            \begin{itemize}
            \item 在统一框架下,实现了三种自由网格耦合算法
            \item 程序可以灵活支持传统单一协调网格和多网格组合两种离散形式
            \item {\bf支持几何和材料大变形功能,流固拉式格式耦合等}
            \item {\color{red}实现“基于部件的组合计算”}
            \end{itemize}
        \end{frame}
        
        \begin{frame}{进展之三:算法改进和性能测试}
            \begin{itemize}
            \item {\bf改进了背景网格耦合算法的数值积分无法达到理论精度的问题}
            \item 改进Step完全拉格朗日算法中的变形信息映射算法
            \item 改进无网格法中邻居点搜索算法
            \item 三种自由网格算法、传统有限元的精度对比
            \end{itemize}
        \end{frame}
 
 
        
%%%%%%%%%%%%%%%%%%%%%%%%%%%%%%%%%%%%%%%%%%%%%%%%%%%%%%%%%%%%%%%%%%%%%%%%%%%%%%%%%%%%%%
%%%%%%%%%%%%%%%%------------IMPLEMENTATION------------------------------------%%%%%%%%
%%%%%%%%%%%%%%%%%%%%%%%%%%%%%%%%%%%%%%%%%%%%%%%%%%%%%%%%%%%%%%%%%%%%%%%%%%%%%%%%%%%%%%        
\begin{frame}
\frametitle{背景网格粘结: 弱形式积分策略(技术难点)}
\begin{itemize}
	\item 直接积分法
	    \begin{itemize}
	        \item 实现简单,但是精度差
	    \end{itemize}
	\item 数学网格切割法
    				\begin{itemize}
    					\item Pros: 精度高
                        \item Cons: 四面体与六面体切割非常耗时
    				\end{itemize}
\end{itemize}
\begin{figure}
		\centering
		\includegraphics[width=3in]{./ljshou_figures/integration.eps}
		%\caption{}
\end{figure}
\begin{block}{}
	\bf \color{red} 如何提高数学网格法切割效率?
\end{block}
\end{frame}

\begin{frame}{四面体与六面体求交算法}
    Key observations:
                \begin{itemize}
                    \item 凸多面体的交一定也是凸多面体
                    \item 凸多面体可以由凸多面体的顶点唯一确定
                    \item 维护面的信息(点的邻接关系)较为耗时
                    \item 并不需要保存交多面体的数据结构,仅仅需要在交上布置高斯积分点
                \end{itemize}
                    
                \begin{block}{Algorithm}
                \begin{enumerate}
                    \item 求出六面体与四面体的交的顶点
                    \item Delaunay三角剖分:根据顶点集合构造四面体网格
                    \item 在小四面体网格中直接布置高斯积分点
                \end{enumerate}
                \end{block}
\end{frame}
	
\begin{frame}{Delaunay三角剖分}
    \begin{itemize}
    \item 三角剖分定义:
        \begin{itemize}
        \item 在数学和计算几何学中,非结构化点集合{\bf P}的三角剖分就是点集的凸包的一个四面体(3D)的网格剖分,并且网格的顶点集合就是点集{\bf P}
        \end{itemize}
    \item 空外接圆(球)性质:
        \begin{itemize}
        \item 三角剖分的任意一个三角形(2D)的外接圆内部不包含其他的顶点
        \end{itemize}
    \end{itemize}
    
    \begin{figure}
    \centering
    \includegraphics[width=2in]{./ljshou_figures/triangulation.png}
    \includegraphics[width=2in]{./ljshou_figures/empty_circum.png}
    \caption{Delaunay三角剖分定义}
    \end{figure}
\end{frame}
	
\begin{frame}{Bowyer-Watson算法}
    \begin{itemize}
    		    \item 增量的算法
    		    \item 利用{\bf空外接圆(球)}的性质
    		    \item 通过在一个初始的三角剖分中,不断地加入新的节点,这种增量迭代的方式,最终得到Delaunay三角剖分
    \end{itemize}
            以二维情况为例,Bowyer-Watson算法流程如下:
            \begin{block}{Bowyer-Watson Algorithm}
                \begin{enumerate}
                    \item 建立一个初始的三角剖分:仅包含一个覆盖住点集{\bf P}的超三角形;
                    \item 依次加入点集{\bf P}中每个点$p_i$:加入新的顶点后,将所有违反空外接圆性质的三角形去掉;去掉不满足空外接圆性质的三角形后,网格会形成一个孔洞,连接点$p_i$与孔洞的顶点;
                    \item 去掉初始的超三角形的边。
                \end{enumerate}
            \end{block}
\end{frame}

\begin{frame}{新的切割算法正确性}
    Proof:
    \begin{itemize}
    		    \item 算法考虑的是两个凸多面体(六面体与四面体)的交,而凸多面体的交一定还是凸多面体
    		    \item 由凸多面体的性质可知,凸多面体的顶点的凸包与该凸多面体是完全等价的
    		    \item Delaunay三角剖分得到的是离散点集的凸包的三角划分
    \end{itemize}
    \begin{block}{}
    \color{red}\bf 以四面体与六面体交的顶点为输入,Delaunay三角剖分算法划分得到的网格正好就是四面体与六面体的交。
        \end{block}
\end{frame}

\begin{frame}{切割算法性能测试}
新的切割算法相比 CGAL(计算几何算法库){\color{red}提高了两个数量级(100倍)},而几乎不损失精度。
	\begin{figure}
		\centering
		\includegraphics[width=2.7in]{./ljshou_figures/tet_cube_test.eps}
		\caption{切割算法性能测试}
	\end{figure}
\end{frame}

\begin{frame}
	\frametitle{背景网格粘结: 弱形式积分策略(技术难点)}
	\begin{itemize}
		\item 剖分仅在材料边界处进行
		\item 内部数学cell直接布高斯积分点
  				\item 极大地减少计算量
	\end{itemize}
	\begin{figure}
			\centering
	    	\includegraphics[width=3in]{./ljshou_figures/mesh_cut.eps}
			\caption{“边界”数学网格与“内部”数学网格}
	\end{figure}
	\begin{block}{}
		\bf \color{blue} 如何快速找到边界数学单元?
	\end{block}
\end{frame}
        
        \begin{frame}{边界数学单元搜索}
            Key Observations:
            	\small
                \begin{itemize}
                \item \textit{"边界数学单元"}一定与\emph{"边界几何单元"}相交
                \item "边界几何单元"可以在O(N)的时间内求出
                \end{itemize}
        		\begin{figure}
		\centering
	   	\includegraphics[width=3.0in]{./ljshou_figures/pysedo_math_cell.eps}
		%\caption{"伪"边界数学单元}
	\end{figure}
	\begin{block}{Algorithm}
		\small
                \begin{enumerate}
                \item 找出\emph{"边界几何单元"}(3D)
                \item 找出与\emph{"边界几何单元"}相交的数学单元
                \item 去掉{\bf“伪”边界数学单元}:通过切割算法,计算数学单元位于材料区域内部的体积
                \end{enumerate}
            \end{block}
        \end{frame}
        
		
\begin{frame}
	\frametitle{大应变计算:变形信息映射}
	\begin{itemize}
		\item 变形信息(例如变形梯度 $F^n$)存储在高斯积分点上
		\item 更新时,需要将变形信息从旧的高斯点映射到新的高斯点上
		\item 张量插值问题(技术难点)
	\end{itemize}
        	\begin{figure}
        		\centering
        		\includegraphics[width=3.5in]{./ljshou_figures/projection.eps}
        		\caption{变形信息映射}
        	\end{figure}
\end{frame}	
	
\begin{frame}
    	\frametitle{大应变计算:变形信息映射}	
    	\begin{itemize}
    		\item 张量插值问题(技术难点)
    		\begin{itemize}
    			\item 基于数学网格节点插值(component-wise)
    			\item 通过张量的特征向量插值
    			\item MLS 进行两套点的插值
    			%\item Strain Smoothing(避免映射)
    		\end{itemize}
    	\end{itemize}
    	\begin{figure}
    		\centering
    		\includegraphics[width=4.0in]{./ljshou_figures/projections_math_node.eps}
    		\caption{基于数学网格节点插值}
    	\end{figure}
    \end{frame}
    
    \begin{frame}{大应变计算:变形信息映射}
        基于移动最小二乘法的张量映射法
        \begin{itemize}
        \item 以新高斯点邻域内的所有的旧高斯积分点为输入,采用 MLS 来构造变形梯度。
        \item 在几何单元节点处保存变形信息。MLS以几何单元的节点构造目标高斯积分点的变形梯度。由于几何单元节点数目大大小于高斯积分点的数目,因此算法效率较高。
        \end{itemize}
        \begin{figure}
    		\centering
    		\includegraphics[width=2.7in]{./ljshou_figures/mls_project.png}
    		\caption{移动最小二乘法构造高斯点处的变形梯度}
    	\end{figure}
    \end{frame}


%%%%%%%%%%%%%%%%%%%%%%%%%%%%%%%%%%%%%%%%%%%%%%%%%%%%%%%%%%%%%%%%%%%%%%%%%%%%%%%%%%%%%%
%%%%%%%%%%%%%%%%%%%%%%%%------------RESULT------------------------------------%%%%%%%%
%%%%%%%%%%%%%%%%%%%%%%%%%%%%%%%%%%%%%%%%%%%%%%%%%%%%%%%%%%%%%%%%%%%%%%%%%%%%%%%%%%%%%%
%\section{算例}
\begin{frame}
\frametitle{背景网格法-收敛性分析}
\begin{itemize}
	\item 泊松方程求解:
	%\begin{equation}
		$-\Delta u = f, \quad in [-1, 1]^3$
	%\end{equation}
	\item 精确解是
	%\begin{equation}
		$u = (1-x^2)(1-y^2)(1-z^2)$
	%\end{equation}
\end{itemize}
	\begin{figure}
	\centering
	\includegraphics[width=3in]{./ljshou_figures/poisson_test.eps}
	\caption{背景网格法收敛性测试}
\end{figure}
\end{frame}

\begin{frame}
\frametitle{背景网格法的数值积分方案与最优收敛}
\begin{itemize}
	\item 在组合几何网格上的数值积分
	\begin{itemize}
		\item 直接积分(图中pGFEM\_1\_1和pGFEM\_1\_2);
		\item 数学网格切割(图中pGFEM\_cut)。
	\end{itemize}
	\item 目前的工作实践证明了\bf{数学网格切割法}是精确、实际可行的方法,确保了方法的理论严密性。
\end{itemize}

\begin{figure}
	\centering
	\includegraphics[width=2.5in]{./ljshou_figures/math_test.eps}
	\caption{背景网格法的数值积分方案测试}
\end{figure}
\end{frame}

\begin{frame}
\frametitle{大变形问题中网格扭曲极端测试-负雅克比网格}
\begin{figure}
	\centering
	\includegraphics[width=3.5in]{./ljshou_figures/jacobian_test.eps}
	\caption{大变形分析中典型的广义有限元网格构型(左).}
	\caption{数学网格法,即使在几何网格在严重扭曲的情况下,仍然可以给出与最规则网格有限元相一致的结果(右). }
\end{figure}
\end{frame}


\begin{frame}{网格生成与计算规模相独立—大规模弱可扩展测试的考虑}
同样针对悬臂梁问题,几何网格由一个四面体网格组成。几何网格主要在于描述求解区域的几何形状和内、外材料边界。插值精度通过调整数学网格尺寸来实现。

\begin{table}[h] 
\scriptsize %此处写字体大小控制命令 
    \begin{tabular}{ccc} 
    \toprule
    Math grid & Geometric grid & Multiple solution\\
    \midrule
    1x1x1	& tet\_mesh\_48531e	& 0.0308048\\
    5x1x1	& tet\_mesh\_48531e	& 0.2257467\\
    10x2x2	& tet\_mesh\_48531e	& 0.2992554\\
    20x4x4	& tet\_mesh\_48531e	& 0.3283354\\
    Ref. solution*	& tet\_mesh\_48531e	& 0.3346891\\
    \midrule
    \multicolumn{3}{c}{*FEM solution obtained using the mesh of 48531 tetrahedral elements} \\
    \bottomrule
    \end{tabular}
\end{table}

    \begin{block}{}
    	\color{red}\bf 背景网格法:几何网格只需生成一次,即可完成各种精度的计算,而无需重新生成网格!(灵活的组合网格计算,网格生成与计算规模独立)
    \end{block}
    
\end{frame}


\begin{frame}{小孔平板剪切—大变形能力测试}
    \begin{itemize}
        \item 采用有限元法、全域无网格法、背景网格法计算平板的大变形
        \item 可压缩Neo-Hooken材料,$\mu=80769.23kN/mm^2$, $\lambda =121153.86kN/mm^2$,泊松比0.3,$E=210000kN/mm^2$
        \item 平板的上下表面简支,下表面中心点固定,上表面施加x轴正方向的均布载荷
    \end{itemize}
\begin{figure}
	\centering
	\includegraphics[width=3.5in]{./ljshou_figures/strip.png}
	\caption{小孔平板网格. (a)变形前;(b)变形后}
\end{figure}
\end{frame}
\begin{frame}{小孔平板剪切—大变形能力测试}
    \begin{itemize}
    \item 参考解是高度加密的有限元解。测试的算法包括背景网格法,全域无网格法和协调网格有限元法三种,均采用Total Lagrange格式
    \end{itemize}

	\begin{figure}
	\centering
	\includegraphics[width=2.5in]{./ljshou_figures/strip.eps}
	\caption{小孔平板问题算法精度对比}
\end{figure}
\end{frame}



\begin{frame}{协调网格悬臂梁弯曲—大变形能力测试}
    \begin{itemize}
    \item 待解问题为一个协调网格悬臂梁。采用有限元法、全域无网格法、背景网格法计算梁的大变形(本例采用协调网格)
    \item 左端全部固定,自由端受均布力荷载。比较各加载步下梁自由端挠度
    \end{itemize}
\begin{figure}
	\centering
	\includegraphics[width=1.3in]{./ljshou_figures/conforming_canti.png}
	%\caption{悬臂梁网格:(a)变形前;(b)变形后}
	\includegraphics[width=2.3in]{./ljshou_figures/conforming_canti.eps}
	\caption{悬臂梁精度对比图}
\end{figure}
\end{frame}



\begin{frame}{组合网格悬臂梁弯曲—组合网格能力测试}
    \begin{itemize}
    \item 待解问题为一个由三个子结构组成的变截面悬臂梁。采用背景网格法,局部无网格法和全域无网格法计算梁的大变形。
    \item 左端全部固定,自由端受均布力荷载。比较各加载步下梁自由端挠度
    \end{itemize}
\begin{figure}
	\centering
	\includegraphics[width=1.3in]{./ljshou_figures/assembly_canti2.png}
	\includegraphics[width=2.3in]{./ljshou_figures/assembly_canti.eps}
	\caption{三种组合网格算法精度的比较}
\end{figure}
\end{frame}



\begin{frame}{核反应堆自重下位移—组合网格有限元应用}
    \begin{figure}
    \centering
    \includegraphics[width=2in]{./ljshou_figures/simulation_model.png}
    \includegraphics[width=1.2in]{./ljshou_figures/simulation_result.png}
    \caption{核反应堆自重下位移}
    \end{figure}
\begin{block}{}
	\color{red}\bf 对于如此复杂的工程结构,生成连续协调的网格变得极其困难!而对于组合网格有限元,网格生成仅花费30分钟。
\end{block}
\end{frame}


%%%%%%%%%%%%%%%%%%%%%%%%%%%%%%%%%%%%%%%%%%%%%%%%%%%%%%%%%%%%%%%%%%%%%%%%%%%%%%%%%%%%%%
%%%%%%%%%%%%%%------------------------ABSTRACT--------------------------%%%%%%%%%%%%%%
%%%%%%%%%%%%%%%%%%%%%%%%%%%%%%%%%%%%%%%%%%%%%%%%%%%%%%%%%%%%%%%%%%%%%%%%%%%%%%%%%%%%%%
%\section{参考文献}
\begin{frame}
\frametitle{参考文献}
\begin{enumerate}
	\tiny
	\item Tian R*, Wu ZD, Wang CW, Scalable FEA on non-conforming assembly mesh. Computer Methods in Applied Mechanics \& Engineering. 2013; 266: 98-111
	\item Tian R*, Meshfree/GFEM in hardware-efficiency prospective. Interaction and multiscale mechanics. DOI:10.12989/imm. 2013.6.2.000, 2013 
	\item Tian R*, Extra-dof-free and linearly independent enrichments in GFEM. Computer Methods in Applied Mechanics \& Engineering. 2013; 266: 1-22
	\item Y. Sudhakar, Wolfgang A. Wall, Quadrature schemes for arbitrary convex/concave volumes and integration of weak form in enriched partition of unity methods. Comput. Methods Appl. Mech. Engrg. 258 (2013) 39–54
	\item Tian R*, Simulation at Extreme-Scale: Co-Design Thinking and Practices. Archive of Computational Methods in Engineering, DOI 10.1007/s11831-014-9095-y, 2014
	\item Rong Tian, Albert C. To, Wing Kam Liu. Conforming local meshfree method. International Journal for Numerical Methods in Engineering 2011; 86(3): 335-357.
	\item Luis Quiroz, Pierre Beckers. Non-conforming mesh gluing in the finite elements method. International Journal for Numerical Methods in Engineering 1995; 38: 2165 – 2184.
	\item Rong Tian, G. Yagawa. Non-matching mesh gluing by meshless interpolation – an alternative to Lagrange multipliers. International Journal for Numerical Methods in Engineering 2007; 71: 473-503.
	\item Powell MJD. The theory of radial basis function approximation in 1990. In Advances in Numerical Analysis, Light FW (ed.), Oxford: Clarendon Press, 1992; 105–203.
	\item J.G. Wang, G.R. Liu. A point interpolation meshless method based on radial basis functions. International Journal for Numerical Methods in Engineering 2002; 54: 1623 – 1648.
	
\end{enumerate}
\end{frame}


%\section{Thanks}
\begin{frame}	
%\Huge{\centerline{Thank You!}}
\begin{block}
	\Huge{\centerline{Thank you}}
	\centerline{shoulinjun@ncic.ac.cn}
\end{block}
\begin{figure}
	\includegraphics[width=2.5in]{./ljshou_figures/UCAS3.png}
\end{figure}
\end{frame}
	
\end{CJK}
\end{document}
